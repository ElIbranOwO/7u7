\documentclass{article}

\usepackage[utf8]{inputenc}
\usepackage[spanish]{babel}
\usepackage{amsmath}

\title{Método Simplex}
\author{María y Emmanuel}

\begin{document}

\maketitle

\section{Introducción}
\label{sec:introduccion}

El método simplex es un algoritmo para resolver problemas de
programación lineal inventado por el matemático norteamericano George
Dantzig en el año 1947.

La forma simplex de un problema de programación lineal es:
Dada una matriz $A$ y vectores $b,c$, maximizar $c^Tx$ sujeto a
$Ax= b$.

\section{Ejemplo}
\label{sec:ejemplo}
Ilustraremos la aplicación del método simplex con un ejemplo.
Considere el siguiente problema:

   \begin{equation*}
   \begin{aligned}
   \text{Maximizar} \quad & 2x_{1}+2x_{2}\\
   \text{sujeto a} \quad &
     \begin{aligned}
      x_{1},x_{2} &\geq  0\\
      2x_{1}+x_{2} &\leq 4\\
      -x_{1}-2x_{2} &\geq -5
     \end{aligned}
   \end{aligned}
 \end{equation*}

 Como en una de las desigualdades aparecen las variables del lado
 izquierdo del simbolo $\geq $ multiplicamos por -1 de ambos lados.

  \begin{equation*}
   \begin{aligned}
   \text{Maximizar} \quad & 2x_{1}+2x_{2}\\
   \text{sujeto a} \quad &
     \begin{aligned}
      x_{1},x_{2} &\geq  0\\
      2x_{1}+x_{2} &\leq 4\\
      x_{1}+2x_{2} &\leq 5
     \end{aligned}
   \end{aligned}
 \end{equation*}

 Lo siguiente por hacer es definir una variable de holgura para cada
 ecuación, dichas variables deben cumplir que sean mayores o iguales
 que 0.
 Por lo tanto definimos $x_3,x_4\geq 0$ de manera que $x_3 = 4-2x_1-x_2$ y
 $x_4 = 5-x_1-2x_2 $.
 Lo cual hace que nuestro problema de programación lineal sea de la
 forma:

  \begin{equation*}
   \begin{aligned}
   \text{Maximizar} \quad & 2x_{1}+2x_{2}\\
   \text{sujeto a} \quad &
     \begin{aligned}
      x_{1},x_{2},x_3,x_4 &\geq  0\\
      2x_{1}+x_{2}+x_3 &= 4\\
      x_{1}+2x_{2}+x_4 &= 5
     \end{aligned}
   \end{aligned}
 \end{equation*}

 A continuación obtenemos un \emph{tablero simplex} despejando las
 variables de holgura.

    \begin{equation*}
     \begin{aligned}
      x_3 &= 4-2x_1-x_2 \\
      x_4 &= 5-x_1-2x_2\\
      0 &\leq x_{1},x_{2},x_3,x_4
     \end{aligned}
 \end{equation*}

Definiendo $z$ igual a nuestra función objetivo, es decir $z =
2x_1+2x_2$, con lo que nuestro tablero es:
 
   \begin{equation*}
     \begin{aligned}
      x_3 &= 4-2x_1-x_2 \\
      x_4 &= 5-x_1-2x_2\\
      \hline
      z &= 2x_1+2x_2
     \end{aligned}
 \end{equation*}

\end{document}

